To generate a Pal\,5-like stream, we ran a direct $N$-body simulation of a low-mass globular cluster, and let it dissolve in a static, Milky Way-like background potential (see Sec.~\ref{ssec:potential}). The model initially consisted of 65,356 single-mass particles of about $0.5\msun$ each, and was evolved for 4\,Gyr using the publicly available code \textsc{Nbody6} \citep{Aarseth03, Nitadori12}. The initial mass of the cluster was $31,090\msun$ and it lost $17,940\msun$ into the tidal tails, leaving the cluster with a present-day mass of $13,150\msun$.

The orbit of the cluster was chosen such that its present-day position, distance and radial velocity match the observed values for Pal\,5. The position in observable coordinates is $RA = 229.0$\,deg, $Dec = -0.1114$\,deg or $l = 0.8521$\,deg, $b = 45.86$\,deg. Its distance was set to $d_{Sun} = 23.2\,pc$ \citep{Harris96}, and the radial velocity to $-58.7$\,km\,s$^{-1}$ \citep{Odenkirchen02}. The proper motion was then chosen such that the profile of the resulting stream matches roughly the observed morphology of Pal\,5 within our choice of galactic potential (Sec.~\ref{ssec:potential}). The values are $\mu_{\alpha\cos(\delta)}=-2.537$\,mas\,yr$^{-1}$ and $\mu_\delta = -2.649$\,mas\,yr$^{-1}$.

The observables were turned into Cartesian coordinates assuming a solar galactocentric distance of 8.33\,kpc and a LSR motion of 239.5\,km/s \citep{Gillessen09}. The solar reflex motion was assumed to be $(11.1, 12.24, 7.25)$\,km\,s$^{-1}$ \citep{Schonrich10}. With these specifications, the present-day Cartesian coordinates of the progenitor in the galactic rest frame are 
\begin{eqnarray}
  x &=& 7816.1\,\mbox{pc}\\
  y &=& 240.02\,\mbox{pc}\\
  z &=& 16640\,\mbox{pc}\\
  vx &=& -37.457\,\mbox{km\,s}^{-1}\\
  vy &=& -151.79\,\mbox{km\,s}^{-1}\\
  vz &=& -21.610\,\mbox{km\,s}^{-1}
\end{eqnarray}

The Challenge can be found on the wiki page of the Gaia Challenge workshop\footnote{\url{http://astrowiki.ph.surrey.ac.uk/dokuwiki/doku.php?id=tests:streams:challenges}}, and we invite everyone to download the Challenge and contribute to this comparison project. The columns are described in the header of the file, and more details can be found on the wiki. The columns give Cartesian coordinates and observables for positions and velocities of all particles. All numbers are either in pc and km/s, or degree and mas/yr, respectively. 



\subsection{The potential}\label{ssec:potential}

The functional form of the potential components is as follows:

Flattened NFW halo:

\begin{eqnarray}
  \Phi_{Halo}(R, z) &=& -\frac{GM}{\sqrt{R^2+\frac{z^2}{q_z^2}}}\ln\left(1+\frac{\sqrt{R^2+\frac{z^2}{q_z^2}}}{R_{Halo}} \right)\\
  M_{Halo} &=& 1.81194\times 10^{12}\msun\\
  R_{Halo} &=& 32260\,pc\\
  q_z &=& 0.8140
\end{eqnarray}

Jaffe bulge:

\begin{eqnarray}
  \Phi_{Bulge} &=& \frac{GM_{Bulge}}{b_{bulge}}\ln{\frac{R}{R+b_{bulge}}}\\
  M_{Bulge} &=& 3.4\times 10^{10}\msun\\
  b_{Bulge} &=& 700.0\,\mbox{pc}
\end{eqnarray}

Miyamoto-Nagai disk:
\begin{eqnarray}
  \Phi_{Disk} &=& -\frac{GM_{Disk}}{\sqrt{R^2+\left(a_{Disk}+\sqrt{z^2+b_{Disk}^2}\right)^2}}\\
  M_{Disk} &=& 1.0\times 10^{11}\,M_{\odot}\\
  a_{Disk} &=& 6500\,pc\\
  b_{Disk} &=& 260\,pc
\end{eqnarray}

\begin{itemize}
  \item $V_C(R_{Sun}) = 249.01\,km/s$
  \item $V_C(R_{Pal5}) = 247.84\,km/s$
  \item $V_C(R_{Halo}) = 251.99\,km/s$
  \item $a(R_{Sun}, 0, 0) = 7.95\,pc/Myr^2$
  \item $a(R_{Pal5}) = a(7816 pc, 240 pc, 16640 pc) = 3.51\,pc/Myr^2$
  \item $a(R_{Halo}, 0, 0) = 2.06\,pc/Myr^2$
\end{itemize}


