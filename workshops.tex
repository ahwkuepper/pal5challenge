The Gaia Challenge workshop series was initiated in 2013 by Justin Read, Daisuke Kawata and Mark Gieles, and the first meeting was held from 19--23 August 2013 at the University of Surrey in Guildford, England. 
The second edition took place from 27--31 October 2014 at the Max-Planck Institute for Astronomy (MPIA) in Heidelberg, Germany. 
Another workshop was held from 31 August to 4 September 2015 at the Universidad de Barcelona, Spain.
The goal of the workshops is to prepare the modeling community for the vast amount of dynamical data that the Gaia satellite is going to provide us with starting in summer 2016 \citep{Perryman01}.

Since the richness of the Gaia data will be valuable for a wide range of modeling problems, the workshops are organized in 5 focus groups:
\begin{enumerate}
\item spherical \& triaxial systems,
\item galactic discs,
\item tidal streams \& galactic halo stars,
\item collisional systems
\item astrophysical parameters 
\end{enumerate}
Each of the groups focusses on a different range of problems, each of which is going to benefit from the high expected quality of the Gaia data. For a detailed description of the individual groups, and for an overview of the activities within the groups, see the Gaia Challenge wiki\footnote{\url{http://astrowiki.ph.surrey.ac.uk/dokuwiki/doku.php}}.