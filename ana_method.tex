For the \textsc{FastForward} method, forward models of streams are generated using the streakline method outlined in \citet{Kupper12} and \citet{Bonaca14}. 
Starting from the present-day position of the cluster, its orbit is integrated backwards in a given trial galaxy potential for a specified time. 
In the case of this test problem, the 3D position and velocity of the cluster is known as well as the integration time of the underlying $N$-body model. 
Hence, these parameters are fixed and only the parameters of the halo potential are sampled from a parameter probablity distribution using the Markov-chain Monte Carlo sampler \textit{emcee} \citep{Foreman13}.

From these initial conditions, the cluster is integrated back to the present-day position, while test particles are released from the instantaneous Lagrange points of the cluster-galaxy system at fixed time intervals of 40\,Myr, resulting in 100 test particles per stream (leading/trailing). 
The positions of the Lagrange points are calculated using the approximation for the Jacobi radius from \citet{King62},
\begin{equation}
r_{L} = \left( \frac{G\,M_c}{\Omega_c^2-\partial^2\Phi/\partial R_c^2}\right)^{1/3},
\end{equation}
where $G$ is the gravitational constant, $M_c$ is the mass of the cluster, $\Omega_c$ is the instantaneous angular velocity of the cluster, and $\partial^2\Phi/\partial R_c^2$ is the second derivative of the galactic potential with respect to the cluster's galactocentric radius, $R_c$.
Test particles are released from $\pm r_L$ along the connecting line between cluster center and galactic center with velocities matching the cluster's angular velocity. 
Random spacial and velocity offsets from these idealized escape conditions are added to the test particles. 
They are drawn from gaussian distributions with dispersions of $0.25r_L$ and 1\,km\,s$^{-1}$, respectively.
 
The resulting, present-day positions and velocities of stream test particles,  are then compared to the mock model particles using a likelihood function of the form
\begin{equation}
L = \prod_j^{N_{data}} \left(\frac{1}{N_{model}}\sum_i^{N_{model}} \exp{\left(-\frac{d_{ij}^2}{2\Delta d^2}\right)} \right).
\end{equation} 
Here, $N_{data}$ is the number of data points of the mock model, $N_{model}$ is the number of model points from the forward model, $d_{ij}$ is the distance of the $i$-th model point from the $j$-th data point in 6D-phase space, and $\Delta d$ is the uncertainty in this distance. 
Since the given data is error-free, hyper-parameters are used to artificially widen the distribution of the points, and create a continuos density distribution. 
This likelihood value is then used as input for the MCMC sampler. 
The chains presented here are the results of 64 ``walkers'', making a total of 200 steps each, after a short burn-in phase of 20 steps.
