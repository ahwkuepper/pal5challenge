To test the basic orbit-fitting algorithm we used 100 points randomly selected from the stream and fit their 
Cartesian coordinates.  In Cartesian space, each data point consists of six phase-space observations.  This means 
that we fit six thickness parameters for a total of nine free parameters.  The resulting MW model PDFs are shown in 
Fig. \ref{plot_nathan_results_Cart}.  The halo flattening is recovered within the 68\% confidence limits, but the orbit
fitting algorithm only recovers the correlation between the  halo mass and scale radius.  The degeneracy in these 
two parameters is due to the orbit-fitting method only being sensitive to the mass within the orbit.  In an NFW halo, 
all models within the mass-radius band indicated in Fig. \ref{plot_nathan_results_Cart} have the same 
mass within the Pal 5 orbit and, as shown in Fig. \ref{plot_nathan_results_Cart_accel}, give the same acceleration at the location 
of the Pal 5 progenitor.  However, this figure also demonstrates that the orbit-fitting algorithm does not quite 
recover the true model as the inferred acceleration differs from the true acceleration significantly.

\begin{figure}
\includegraphics[width=83mm]{./figures/Nathan_results_CartAll.pdf}
  \caption{The MW model PDFs inferred using the orbit-fitting algorithm on Cartesian stream data.  The red and blue 
  contours enclose the 68\% and 95\% confidence regions, the star and red vertical lines indicate 
  the generative values, and the grey shaded regions correspond to areas of increasing 
  probability.  (Sec.~\ref{ssec:nathan_results})}
  \label{plot_nathan_results_Cart}
\end{figure}

\begin{figure}
\includegraphics[width=83mm]{./figures/Nathan_results_Accel_CartAll.pdf}
  \caption{The inferred acceleration at the location of the Pal-5 progenitor using the 
  orbit-fitting algorithm with Cartesian data.  The red line indicates the generative value.  (Sec.~\ref{ssec:nathan_results})}
  \label{plot_nathan_results_Cart_accel}
\end{figure}


A slightly more realistic test is to utilize observational data instead of Cartesian data.  Observational data differs from Cartesian 
data in two key respects.  Firstly, it reduces the dimensionality of the stream fitting to five phase-space coordinates as 
we use the angular distance on the sky between model stream points and data points as the constraint.  Secondly, it introduces 
a constraint on the local circular speed as the observations of the stream's radial velocity and proper motions are 
convolved with $v_{c}(R_{\circ})$.  The resulting PDFs of the model parameters and acceleration at the location of the 
Pal 5 progenitor are shown in Figs. \ref{plot_nathan_results_ObsData}-\ref{plot_nathan_results_ObsData_accel}.
These figures show that the use of observational data improves the fits greatly.  The convolution of the observed
motion with the local circular speed breaks the halo mass-scale radius degeneracy.  The generative model is recovered, as 
is the acceleration at the location of Pal 5.


\begin{figure}
\includegraphics[width=83mm]{./figures/Nathan_results_ObsData.pdf}
  \caption{The MW model PDFs inferred using the orbit-fitting algorithm on Cartesian stream data.  The red and blue 
  contours enclose the 68\% and 95\% confidence regions, the star and red vertical lines indicate 
  the generative values, and the grey shaded regions correspond to areas of increasing 
  probability.  (Sec.~\ref{ssec:nathan_results})}
  \label{plot_nathan_results_ObsData}
\end{figure}

\begin{figure}
\includegraphics[width=83mm]{./figures/Nathan_results_Accel_ObsData.pdf}
  \caption{The inferred acceleration at the location of the Pal-5 progenitor using the 
  orbit-fitting algorithm with Cartesian data.  The red line indicates the generative value.  (Sec.~\ref{ssec:nathan_results})}
  \label{plot_nathan_results_ObsData_accel}
\end{figure}

It is worth noting that these tests are still not completely realistic.  The progenitor has remained fixed in Cartesian space, no 
observational error has been included, and the majority of the model parameters were not explored.  While the results shown in 
Figs. \ref{plot_nathan_results_ObsData}-\ref{plot_nathan_results_ObsData_accel} are promising, it is not clear how well 
orbit-fitting will perform in more realistic tests.