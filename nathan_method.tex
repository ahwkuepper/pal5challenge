
The orbit-fitting method assumes that the stream is generated by the orbit of a single particle in a fixed potential.  
Due to this method's simplicity, it is the most efficient stream modelling method.  We use a modified version of the 
algorithm described in \citet{Deg13}.  The main difference is that the likelihood is calculated for a data point being drawn 
from the entire \textit{stream} rather than from the nearest stream point.  Explicitly, the likelihood of particular 
data point, $D_{j}$ being drawn from a model stream, $\mathbf{M}$, is
is 
\begin{equation}
 \mathcal{L}(D_{j}|\mathbf{M})=\sum_{i=1}^{N}\mathcal{L}(D_{j}|M_{i})~,
\end{equation}
where $N$ is the number of points comprising the stream and $M_{i}$ is the $i$-th orbital point. The likelihood of being 
drawn from each model stream point is  
\begin{equation}
 \mathcal{L}(D_{j}|M_{i})=\Pi_{k}^{N_{k}}\frac{1}{\sqrt{2\pi\sigma_{k}^{2}}}
 e^{-\frac{(D_{j,k}-M_{i,k})^{2}}{2\sigma_{k}^{2}}}~,
\end{equation}
where $N_{k}$ is the number of phase-space observations of the data point, and $\sigma_{k}$ is the combined error and thickness 
associated with that phase-space observation.  Any stream has some thickness or dispersion in each dimension and every observation 
has some error.  In this test case we do not include any observational error but we do fit for a uniform thickness in each dimension.
This introduces a number of additional free parameters.

For simplicity we model the MW using the generative Jaffe bulge, Miyamoto-Nagai disk, and flattened NFW halo with 
all parameters fixed except the halo mass, scale length, and flattening.  To recover the PDF, we performed a Bayesian analysis with logarithmic priors on all 
MW model free parameters using the \textit{emcee} algorithm with 200 walkers each run for 2000 steps.
